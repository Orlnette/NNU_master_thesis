%第一章

\section{绪论}
\subsection{简单介绍}
\subsubsection{插入表格}
需要分段的文字换行两次就可以分隔开了,\LaTeX 会帮助你整理格式。

你看如果这样就能分三段了,如果需要加粗可以选中要\textbf{加粗的字},按ctrl+b快捷加粗。下划线则需要自己键入指令:\underline{下划线}。
\paragraph{使用四级标题}

\subsubsection{插入表格}
可以直接通过将图片粘贴到这里,选择上传到的文件夹,修改图片名称即可。可以直接拖拽文件到文件夹,点确定就行了,图片都会在云端存储,不用担心,也可以跨项目上传。在文中引用图片使用\ref{fig:exm}命令即可。
\begin{figure}[h]
    \centering
    \includegraphics[width=0.3\linewidth]{figs/image.png}
    \caption{bjh好美}
    \label{fig:exm}
\end{figure}

\subsubsection{插入图表}如表\ref{tab:example}所示,表格也可以去在线生成,或者复制Excel里的格式扔给gpt啥的生成\LaTeX 代码就行。

\begin{table}[h]
\centering
\caption{示例表格}
\label{tab:example}
\renewcommand{\arraystretch}{1.3} % 设置行高为1.3倍,默认是1.0,有需要的添加上去改
\begin{tabular}{ccc}%有几个c代表有几列,c代表居中,l和r分别表示靠左或靠右。
\toprule
列1 & 列2 & 列3 \\%每列的数据由&分隔。
\midrule
数据1 & 数据2 & 数据3 \\
数据4 & 数据5 & 数据6 \\
\bottomrule
\end{tabular}
\end{table}

\subsubsection{插入公式}公式如果带编号则需要包裹在下面的环境中,\LaTeX 公式编辑建议去网上搜在线的生成器,或者Axmath这种软件,很方便的,但用熟练的话就能自己打了,我随便打一个吧。
\begin{equation}
    X\left[ k \right] =\sum_{m=0}^{M-1}{x\left[ m \right] \left( \cos \left( \frac{2\pi}{M}km \right) -\ j\sin \left( \frac{2\pi}{M}km \right) \right)},
\end{equation}
如果是多组公式的话可以用aligned包裹,或者挨个equation就行了。举个栗子,注意各个公式之间的换行符。%注意这里没和上面的公式隔开一行,所以不是新的段落,没有首行缩进字符。

\begin{equation}
    \begin{aligned}
        X\left[ k \right] =\sum_{m=0}^{M-1}{x\left[ m \right] \left( \cos \left( \frac{2\pi}{M}km \right) -\ j\sin \left( \frac{2\pi}{M}km \right) \right)},\\
        X\left[ k \right] =\sum_{m=0}^{M-1}{x\left[ m \right] \left( \cos \left( \frac{2\pi}{M}km \right) -\ j\sin \left( \frac{2\pi}{M}km \right) \right)},\\
        X\left[ k \right] =\sum_{m=0}^{M-1}{x\left[ m \right] \left( \cos \left( \frac{2\pi}{M}km \right) -\ j\sin \left( \frac{2\pi}{M}km \right) \right)},
    \end{aligned}
\end{equation}

关于正文中如果想要插入公式的话需要dollar符号包裹,比如$ conv\ 3 \times 3$,也可以用\( \),空格通常需要以$\ $形式出现。

\subsubsection{参考文献}需要去找文献的bibtex格式,把复制到paper.bib这个文件里面,然后在文中\cite{nussbaumer1982fast}即可,默认的引用就是上引,不需要再修改了。后面的参考文献也会自动生成。

\subsubsection{小标题不会自动换行}ctrl+s或者ctrl+enter进行编译,有问题就多编译几次,右边会生成pdf的预览,随时关注生成的效果。


\newpage%分页指令